%!TEX TS-program = xelatex

% Шаблон документа LaTeX создан в 2018 году
% Алексеем Подчезерцевым
% В качестве исходных использованы шаблоны
% 	Данилом Фёдоровых (danil@fedorovykh.ru) 
%		https://www.writelatex.com/coursera/latex/5.2.2
%	LaTeX-шаблон для русской кандидатской диссертации и её автореферата.
%		https://github.com/AndreyAkinshin/Russian-Phd-LaTeX-Dissertation-Template

\documentclass[a4paper,14pt]{article}

\input{data/preambular.tex}
\begin{document} % конец преамбулы, начало документа
    \input{data/title.tex}
    \tableofcontents
    \pagebreak


    \section{Задание}

    Бригада №5.
    
    MAX 10 Plus

    \begin{enumerate}
        \item задание
    \end{enumerate}


    \section{Выполнение работы}

    \subsection{Задание №5}

    Была выполнена программа examples/tutorials/systemverilog/dpi_basic/.
    Результат представлен на рис.~\ref{05_wave}.
    Сначала включается цвет по умолчанию -- красный, затем включается зеленый, после управление передается в С код.
    В C коде печатается сообщение, включается желтый свет, вызывается функция $sv\_WaitForRed$, которая ожидает 10нс, 
    после включается красный свет, управление возвращается в SystemVerilog.
    Через 10нс снова включается зеленый.

    Фрагмент SystemVerilog теста.
    {\small \VerbatimInput{code/05_test.sv}}

    Фрагмент C кода.
    {\small \VerbatimInput{code/05_foreugn.c}}

    \begin{figure}[H]
        \centering
        \includegraphics[width=\linewidth]{images/05_wave}
        \caption{Вейвформа для dpi_basic}
        \label{fig:05_wave}
    \end{figure}
    
	\subsection{Основное задание}
	
	
	\subsection{Дополнительное задание}
	

    \section{Исходные коды}

    Исходные коды доступны на \href{https://github.com/AsciiShell/hse_hlimds_labs}
    {https://github.com/AsciiShell/hse\_hlimds\_labs}.

    Pull request работы \href{https://github.com/AsciiShell/hse_hlimds_labs/pull/4}
    {https://github.com/AsciiShell/hse\_hlimds\_labs/pull/4}.


    \section{Выводы по работе}

\end{document} % конец документа
