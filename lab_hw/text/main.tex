%!TEX TS-program = xelatex

% Шаблон документа LaTeX создан в 2018 году
% Алексеем Подчезерцевым
% В качестве исходных использованы шаблоны
% 	Данилом Фёдоровых (danil@fedorovykh.ru) 
%		https://www.writelatex.com/coursera/latex/5.2.2
%	LaTeX-шаблон для русской кандидатской диссертации и её автореферата.
%		https://github.com/AndreyAkinshin/Russian-Phd-LaTeX-Dissertation-Template

\documentclass[a4paper,14pt]{article}

\input{data/preambular.tex}
\begin{document} % конец преамбулы, начало документа
    \input{data/title.tex}
    \tableofcontents
    \pagebreak


    \section{Задание}

	Подключение камеры D8 к плате с ПЛИС. 
	Необходимо подключить модуль камеры D8M к плате ПЛИС и реализовать систему фотографирования (получение фотографий, преобразование в файл PNG или другой графический формат и передачу на компьютер через UART соединение (USB).

    \section{Выполнение работы}

	В качестве камеры использовался блок D8M-GPIO (рис.~\ref{fig:d8m}), который устанавливается в 40-пиновый слот расширения платы DE0-Nano (рис.~\ref{fig:de0}). 
	Блок камеры основан на чипе OV8865, который позволяет получать изображение до 3268x2448 и до 60Гц (при пониженном разрешении). 
	Блок камеры имеет высокоскоростной интерфейс MIPI, который поддерживает 10-битное параллельную передачу данных.
	Управление блоком происходит через интерфейс I2C.
	Камере можно задавать разрешение, частоту кадров, управлять фокусным расстоянием камеры.
	
	\begin{figure}[H]
		\centering
		\includegraphics[width=0.5\linewidth]{images/d8m}
		\caption{Камера D8M}
		\label{fig:d8m}
	\end{figure}
	
	\begin{figure}[H]
		\centering
		\includegraphics[width=0.7\linewidth]{images/de0-nano}
		\caption{Плата DE0-Nano}
		\label{fig:de0}
	\end{figure}
	
	В программной части используется NIOS2 ядро (рис.~\ref{fig:nios}), которое управляет камерой и декодером через интерфейс I2C. 
	
 	\begin{figure}[H]
		\centering
		\includegraphics[width=\linewidth]{images/nios}
		\caption{Схема проекта}
		\label{fig:nios}
	\end{figure}
	
	Матрица камеры состоит из множества чувствительных элементов, каждый из которых воспринимает только свой цвет (рис.~\ref{fig:matrix}). 
	Изображение в таком формате является RAW изображением, занимает много памяти и требует дополнительной обработки для естественного вида.
	Для этого выполняется цветокоррекция, которая учитывает каждый цвет с определенным весом и усредняет их для каждого пикселя.
	
	\begin{figure}[H]
		\centering
		\includegraphics[width=0.7\linewidth]{images/matrix}
		\caption{Схема расположения сенсоров камеры}
		\label{fig:matrix}
	\end{figure}
	
	
	Взаимодействие с ПК осуществляется с помощью модуля jtag\_uart.
	С помощью данного модуля выполняется управление параметрами камеры и передача изображения между платой DE0-Nano и ПК.
	
	Со стороны ПК используется библиотека jtag\_atlantic пакета Quartus, которая отвечает за передачу данных с модулем jtag uart на плате.
	С помощью данной библиотеки можно управлять камерой и получать с нее изображение.
	
    

    \section{Исходные коды}

    Исходные коды доступны на \href{https://github.com/AsciiShell/hse_hlimds_labs}
    {https://github.com/AsciiShell/hse\_hlimds\_labs}.

    Pull request работы \href{https://github.com/AsciiShell/hse_hlimds_labs/pull/6}
    {https://github.com/AsciiShell/hse\_hlimds\_labs/pull/6}.

\end{document} % конец документа
