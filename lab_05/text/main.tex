%!TEX TS-program = xelatex

% Шаблон документа LaTeX создан в 2018 году
% Алексеем Подчезерцевым
% В качестве исходных использованы шаблоны
% 	Данилом Фёдоровых (danil@fedorovykh.ru) 
%		https://www.writelatex.com/coursera/latex/5.2.2
%	LaTeX-шаблон для русской кандидатской диссертации и её автореферата.
%		https://github.com/AndreyAkinshin/Russian-Phd-LaTeX-Dissertation-Template

\documentclass[a4paper,14pt]{article}

\input{data/preambular.tex}
\begin{document} % конец преамбулы, начало документа
    \input{data/title.tex}
    \tableofcontents
    \pagebreak


    \section{Задание}

    Бригада №5.

    MAX 10 NEEK	

    \begin{enumerate}
        \item Изучить примеры и документацию на проектирование многопроцессорных систем;
        \item Портировать проект Cyclone10LP\_multiprocessor на плату;
        \item Разработать многопроцессорную систему на 9 ядер и соединить их в сеть.
    \end{enumerate}


    \section{Выполнение работы}

	Был изучен пример Cyclone10LP\_multiprocessor.
	Результат компиляции представлен на рис.~\ref{fig:main_cyclone}.
	
	Далее в проект были внесены изменения для работы с платой MAX 10 NEEK.
	Результат компиляции на рис.~\ref{fig:main_max10}.
	
    \begin{figure}[H]
		\centering
		\includegraphics[width=\linewidth]{images/main_cyclone}
		\caption{Результат компиляции проекта}
		\label{fig:main_cyclone}
	\end{figure}

    \begin{figure}[H]
		\centering
		\includegraphics[width=\linewidth]{images/main_max10}
		\caption{Результат портирования проекта на MAX10}
		\label{fig:main_max10}
	\end{figure}

    \section{Самостоятельная работа}

   %   {\small \VerbatimInput{code/04_extra.txt}}


    \section{Исходные коды}

    Исходные коды доступны на \href{https://github.com/AsciiShell/hse_hlimds_labs}
    {https://github.com/AsciiShell/hse\_hlimds\_labs}.

    Pull request работы \href{https://github.com/AsciiShell/hse_hlimds_labs/pull/5}
    {https://github.com/AsciiShell/hse\_hlimds\_labs/pull/5}.


    \section{Выводы по работе}

    В ходе работы был изучены технологии создания многопроцессорных систем и взаимодействия между различными ядрами.

\end{document} % конец документа
