%!TEX TS-program = xelatex

% Шаблон документа LaTeX создан в 2018 году
% Алексеем Подчезерцевым
% В качестве исходных использованы шаблоны
% 	Данилом Фёдоровых (danil@fedorovykh.ru) 
%		https://www.writelatex.com/coursera/latex/5.2.2
%	LaTeX-шаблон для русской кандидатской диссертации и её автореферата.
%		https://github.com/AndreyAkinshin/Russian-Phd-LaTeX-Dissertation-Template

\documentclass[a4paper,14pt]{article}

\input{data/preambular.tex}
\begin{document} % конец преамбулы, начало документа
    \input{data/title.tex}
    \tableofcontents
    \pagebreak


    \section{Задание}

    Бригада №5.

    \begin{enumerate}
        \item Запустить реализацию MobileNet на DE0-Nano;
        \item Заменить камеру OV7670 на D8M.
    \end{enumerate}


    \section{Выполнение работы}

    \subsection{Установка зависимостей}

	Для работы была установлена последняя версия python 3.5, имеющая собранные бинарные коды.
	Далее была выполнена установка зависимостей проекта с указанием версий для ключевых пакетов.
	
	{\small \VerbatimInput{../requirements.txt}}

	\subsection{Основное задание}
	
	Было выполнено обучение нейронной сети, подобраны параметры генерации весов, созданы необходимые verilog файлы.
	После итоговый проект был собран и загружен на плату.
	
	\subsection{Дополнительное задание}
	
	Была выполнена миграция на камеру D8M. 
	Это камера, в основе которой лежит камера OV8865.
	Полученный проект был загружен на плату и протестирован.

    \section{Исходные коды}

    Исходные коды доступны на \href{https://github.com/AsciiShell/hse_hlimds_labs}
    {https://github.com/AsciiShell/hse\_hlimds\_labs}.

    Pull request работы \href{https://github.com/AsciiShell/hse_hlimds_labs/pull/3}
    {https://github.com/AsciiShell/hse\_hlimds\_labs/pull/3}.


    \section{Выводы по работе}
    В ходе работы была обучена нейронная сеть MobileNet для распознавания цифр.
    Итоговая нейросеть была сконвертирована на плату DE0-Nano и запущена с подключенной камерой и LCD дисплеем, на который выводилось изображение.
    Дополнительно была проведена работа по замене камеры на D8M.

	\pagebreak
	\begin{landscape}
	\begin{spacing}{15}
		{\fontsize{200}{220}\selectfont 1 \hfill 2 \hfill 3 \hfill 4 \hfill 5 
			
		 6 \hfill 7 \hfill 8 \hfill 9 \hfill 0}
	\end{spacing}
	\end{landscape}


\end{document} % конец документа
